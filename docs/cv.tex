\documentclass[letterpaper,10pt]{article} % Default font size and paper size

\usepackage{times}
\usepackage{array}
\newcolumntype{L}[1]{>{\raggedright\let\newline\\\arraybackslash\hspace{0pt}}m{#1}}
\newcolumntype{C}[1]{>{\centering\let\newline\\\arraybackslash\hspace{0pt}}m{#1}}
\newcolumntype{R}[1]{>{\raggedleft\let\newline\\\arraybackslash\hspace{0pt}}m{#1}}

\usepackage{hyperref} % Required for adding links and customizing them

\usepackage[letterpaper,ignoreall,top=1.75cm,bottom=1.75cm,%
left=2cm,right=2cm,foot=0cm,head=0cm]{geometry}

\usepackage{titlesec} % Used to customize the \section command
\titleformat{\section}{\Large\scshape\raggedright}{}{0em}{}[\titlerule] % Text formatting of sections
\titlespacing{\section}{0pt}{3pt}{3pt} % Spacing around sections

\begin{document}

\pagestyle{empty} % Removes page numbering

%----------------------------------------------------------------------------------------
%   NAME AND CONTACT INFORMATION
%----------------------------------------------------------------------------------------

\section{\Huge \textbf{ Zackary Falls}}
\noindent 
\begin{tabular}{R{1.7cm}l}

\textsc{Address:} & 806 Ashland Avenue, Buffalo, New York 14222 \\
\textsc{Phone:} & +1 (315) 806-1955\\
\textsc{email:} & \href{mailto:zfalls@me.com}{zfalls@me.com} \\
\multicolumn{2}{c}{} \\

\end{tabular}



%----------------------------------------------------------------------------------------
%   EDUCATION
%----------------------------------------------------------------------------------------

\section{Education}
\noindent 
\begin{tabular}{R{3cm}|p{13cm}} 

\textsc{2012 -- Present} & Doctor of Philosophy (Ph.D.) in \textsc{Computational Chemistry} \\ 
& \textbf{University at Buffalo, State University of New York}, Buffalo, New York \\
& \hspace{3mm} \small Thesis: ``Elucidating Chemical Structures via DFT Investigations'' \\
& \hspace{3mm} \small | Advisor: Prof. Eva Zurek \\
\multicolumn{2}{c}{} \\

%------------------------------------------------

\textsc{2008 -- 2012} & Bachelor of Science (B.S.) in \textsc{}\textsc{Chemistry} -- ACS Accredited \\
& \normalsize\textbf{Canisius College}, Buffalo, New York\\
\multicolumn{2}{c}{} \\

\end{tabular}



%----------------------------------------------------------------------------------------
%   RESEARCH EXPERIENCE 
%----------------------------------------------------------------------------------------

\section{Research Experience}
\noindent 
\begin{tabular}{R{3cm}|p{13cm}} 

\emph{Current} & Graduate Researcher at \textbf{University at Buffalo, State University of New York} \\
\textsc{2012 -- Present} & \hspace{3mm} \small | Advisor: Eva Zurek \\ 
& \footnotesize{Primary research topic involves the molecular modeling of homogeneous and heterogeneous polyolefin polymerization catalyzed by single-site metallocene complexes. Metallocenes need to be activated by a co-catalyst such as methylaluminoxane (MAO) in order for polymerization to occur. The structure(s) of MAO have remained a mystery despite several experimental and theoretical studies. Computational methods are employed to explore the dynamic equilibria of various plausible MAO oligomers and structural entities for this elusive, yet significant, co-catalyst. We are continuing to study the interaction of MAO oligomers with MgCl$_{2}$ support. Our secondary project involves further development of, \textsc{XtalOpt}, an open source evolutionary algorithm for crystal structure prediction.}\\
\multicolumn{2}{c}{} \\

%------------------------------------------------

\textsc{2011 -- 2012} & Undergraduate Researcher at \textbf{Canisius College} \\
& \hspace{3mm} \small | Advisor: Jeremy Steinbacher \\ 
& \footnotesize{Research in the field of bio-organic material synthesis, specifically mesoporous silica nanoparticles. Qualitative and quantitative analyses were employed for these products using thermogravimetric analysis, thin-layer chromatography, nuclear magnetic resonance, and other methods. Employed techniques to synthesize functionalized polyhedral oligomeric silsesquioxanes.}\\
\multicolumn{2}{c}{} \\

%------------------------------------------------

\textsc{Summer 2011} & Undergraduate Researcher at \textbf{University at Buffalo, State University of New York} \\
& \emph{Research Education for Undergraduates} \\
& \hspace{3mm} \small | Advisor: Eva Zurek \\ 
& \footnotesize{Ten week program to allow for the experience of graduate level research as an undergraduate. Research focused on testing a newly written random docking algorithm to screen a library of possible monomers used for molecularly imprinted polymers/xerogels.}\\
\multicolumn{2}{c}{} \\

\end{tabular}



%----------------------------------------------------------------------------------------
%   PUBLICATIONS
%----------------------------------------------------------------------------------------

\section{Publications}
\noindent

\begin{itemize}

    \item \textbf{Falls, Z.}; Lonie, D. L.; Avery, P.; Shamp, A.; Zurek, E. ``\textsc{XtalOpt} version r9: An open–source evolutionary algorithm for crystal structure prediction" Comp. Phys. Comm. 2015, \emph{In Press}. doi: 10.1016/j.cpc.2015.09.018

    \item Shamp, A.; Terpstra, T.; Bi, T.; \textbf{Falls, Z.}; Avery, P.; Zurek, E. ``Decomposition Products of Phosphine Under Pressure: PH$_{2}$ Stable and Superconducting?", 2015, \emph{Preprint}. arXiv:1509.05455.
        
    \item \textbf{Falls, Z.}; Tyminska, N.; Zurek, E. ``The Dynamic Equilibrium Between (AlOMe)$_{n}$ Cages and (AlOMe)$_{n}\cdot$(AlMe$_{3}$)$_{m}$ Nanotubes in Methylaluminoxane (MAO): A First-Principles Investigation", Macromolecules. 2014, 47 (24), 8556–8569. doi: 10.1021/ma501892v

    \item Wach, A.; Chen, J.; \textbf{Falls, Z.}; Lonie, D.; Mojica, E.; Aga, D.; Autschbach, J.; Zurek, E. ``Determination of the Structures of Molecularly Imprinted Polymers and Xerogels Using an Automated Stochastic Approach", Anal. Chem. 2013, 85 (18), 8577-8584. doi: 10.1021/ac402004z

\end{itemize}



%----------------------------------------------------------------------------------------
%   TEACHING EXPERIENCE 
%----------------------------------------------------------------------------------------

\section{Teaching Experience}
\noindent 
\begin{tabular}{R{3cm}|p{13cm}} 

\emph{Current} & Teaching Assistant at \textbf{University at Buffalo, State University of New York}\\
\textsc{2012 -- Present} & \small General Chemistry\\
\multicolumn{2}{c}{} \\

%------------------------------------------------

\textsc{Spring 2012} & Teaching Assistant at \textbf{Canisius College}\\
& \small General Chemistry\\
\multicolumn{2}{c}{} \\

%------------------------------------------------

\textsc{Fall 2011} & Teaching Assistant at \textbf{Canisius College}\\
& \small Analytical Chemistry\\
\multicolumn{2}{c}{} \\

\end{tabular}



%----------------------------------------------------------------------------------------
%   SCHOLARSHIPS AND ADDITIONAL INFO
%----------------------------------------------------------------------------------------

\section{Awards Received}
\noindent 
\begin{tabular}{R{3cm}|p{13cm}} 

2012 -- 2015 & Gordon Harris Fellowship Award \\
& \textbf{University at Buffalo, State University of New York} \\
\multicolumn{2}{c}{} \\

%----------------------------------------------------------------------------------------

2013 & Graduate Student Employee’s Union Professional Development Award \\
& \textbf{University at Buffalo, State University of New York} \\
\multicolumn{2}{c}{} \\

%----------------------------------------------------------------------------------------

2012 -- 2013 & Marjorie Winkler Fellowship Award \\
& \textbf{University at Buffalo, State University of New York} \\
\multicolumn{2}{c}{} \\

%----------------------------------------------------------------------------------------

2012  & Merck Index Award \\
& \textbf{Canisius College} \\
\multicolumn{2}{c}{} \\

%----------------------------------------------------------------------------------------

2012  & REU Chemistry Leadership Award \\
& \textbf{National Science Foundation} \\
\multicolumn{2}{c}{} \\

\end{tabular}



%----------------------------------------------------------------------------------------
%   CONFERENCES ATTENDED
%----------------------------------------------------------------------------------------

\section{Conferences Attended}
\noindent 
\begin{tabular}{R{3cm}|p{13cm}} 

\textsc{July} 2014 & Gordon Research Conference -- \textbf{Atomic and Molecular Interactions} \\
& \small{\emph{``Exploring the Dynamic Equilibrium between MAO (methylaluminoxane) Oligomers via First Principles Calculations"}} \\
& \small Poster \\
\multicolumn{2}{c}{} \\

%------------------------------------------------

\textsc{May} 2014 & Graduate Student Symposium -- \textbf{University at Buffalo, State University of New York} \\
& \small{\emph{``Exploring the Dynamic Equilibrium between MAO (methylaluminoxane) Oligomers via First Principles Calculations"}} \\
& \small Presentation \\
\multicolumn{2}{c}{} \\

%------------------------------------------------

\textsc{May} 2013 & Canadian Chemistry Conference -- \textbf{Canadian Society for Chemistry} \\
& \small{\emph{``Computations of the Equilibria between various MAO, (AlOMe)$_n$, Oligomers and their EFG Tensors"}} \\
& \small Poster \\
\multicolumn{2}{c}{} \\

%------------------------------------------------

\textsc{May} 2013 & Graduate Student Symposium -- \textbf{University at Buffalo, State University of New York} \\
& \small{\emph{``Analysis of Electric Field Gradient Tensors at the Quadrupolar Aluminum Nuclei for Oligomers of Methylaluminoxane"}} \\
& \small Poster \\
\multicolumn{2}{c}{} \\

%------------------------------------------------

\textsc{October} 2012 & American Chemical Society -- \textbf{Northeast Regional Meeting} \\
& \small{\emph{``Interactions in Cp$_2$ZrMe$_2$--catalyzed, MAO (methylaluminoxane) Catalyzed Heterogeneous Polymerization: A Computational Approach"}} \\
& \small Poster \\
\multicolumn{2}{c}{} \\

%------------------------------------------------

\textsc{March} 2012 & American Chemical Society -- \textbf{National Meeting} \\
& \small{\emph{``Computational Analysis of Imprinting Polymers and Xerogels using a Random Docking Program"}} \\
& \small Poster \\
\multicolumn{2}{c}{} \\


\end{tabular}


%----------------------------------------------------------------------------------------
%   COMPUTER SKILLS 
%----------------------------------------------------------------------------------------

%\section{Computer Skills}
%\noindent 
%\begin{tabular}{R{3cm}|p{13cm}} 

%Basic Knowledge: & \textsc{php}, my\textsc{sql}, \textsc{html}, Access, \textsc{Linux}, ubuntu\\

%Intermediate Knowledge: & \textsc{vba}, Excel, Word, PowerPoint\\

%\end{tabular}



\end{document}
